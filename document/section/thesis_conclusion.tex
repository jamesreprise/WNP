\section{Conclusion}
\label{sec:conclusion}

In this document we have introduced wireless sensor mesh networks and the importance of node placement in those networks. Having given a mathematical definition of the problem and a practical definition in the form of the indoor museum example and detailing how constraints may affect our placement of the nodes, we discuss various optimisation methods both traditional and novel, implement them and compare their performance. The features of the software utilising these methods have been demonstrated and comparisons of several different measures of performance have been carried out. Potential extensions to this document and the software have been discussed alongside questions left open.
\subsection{Contributions} 
	\label{sec:conclusion_contribs} 
	
		\begin{description}	
		
			\item[$\bullet$ A software library for the optimising of 2D node placement]\hfill
			
			This library is general enough to be used in any 2D node placement project and also easily modifiable to suit more specific needs.
			
			\item[$\bullet$ Software aiding the use of the optimisation library]\hfill
			
			For users who do not have a great deal of programming skill, this software works out of the box to interface with the library in an easy to understand way.
			
			\item[$\bullet$ This document]\hfill
						
			We review techniques and compare their usefulness in response to the growth of the number of dimensions and the complexity of the objective function (see chapter \ref{sec:problem}).
			
		\end{description}