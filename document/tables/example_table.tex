% It's often a good idea to generate the LaTeX code for tables (python script or similar) so that if you rerun your code you don't have to typeset your results again by hand!
\begin{table}[H]
	\centering
	\scriptsize
	\caption[A demonstration of a table typeset in LaTeX.]{An example of a table formatted with caption.}
	
	% Tune the following two values that are being multiplied by the variable \textwidth
	% to control how large the scale of the table is, and how much is is squashed back 
	% to the final size.	
	\resizebox{0.8\textwidth}{!}{
		\begin{tabularx}{0.46\textwidth}{c|c|c|c|c|c}
			\toprule
			Some & Relevant & Fields & From & Your & Data\\
			\midrule
			0 & 0 & 0 & 0 & 0 & 0\\
			1 & 1 & 1 & 1 & 1 & 1\\
			2 & 2 & 2 & 2 & 2 & 2\\
			\bottomrule
		\end{tabularx}
		\label{tbl:example_table}
	}
\end{table}

